\documentclass{article}
\usepackage{amsfonts}
\usepackage{amsmath}
\usepackage{amssymb}
\usepackage{amsthm}
\usepackage{bm}
\usepackage{booktabs}
\usepackage{graphicx} % Required for inserting images
\usepackage{indentfirst}
\usepackage{mathtools}
\usepackage{multirow}
\usepackage[round]{natbib}
%\usepackage[square,sort,comma,numbers]{natbib}

\usepackage[top=2.5cm, bottom=2.5cm, left=2.5cm, right=2.5cm]{geometry}
\usepackage{titlesec}

\newtheorem{theorem}{Theorem}[section]
\newtheorem{definition}{Definition}[section]
\newtheorem{example}{Example}
\newtheorem{proposition}{Proposition}
% \newtheorem*{corollary}{Corollary}

\DeclareMathOperator{\atan}{atan}
\DeclareMathOperator{\Exp}{Exp}
\DeclareMathOperator{\Log}{Log}
\newcommand{\norm}[1]{\left\lVert#1\right\rVert}

% Define \mathpzc font lower case, \mathcal is for upper case
\DeclareFontFamily{OT1}{pzc}{}
\DeclareFontShape{OT1}{pzc}{m}{it}{<-> s * [1.10] pzcmi7t}{}
\DeclareMathAlphabet{\mathpzc}{OT1}{pzc}{m}{it}

\makeatletter
\def\x{\@ifnextchar_{\xsub}{\mathpzc{x}}} % Default \x to \mathpzc{x}
\def\xsub_#1{\mathpzc{x}_{\mkern4mu #1}}  % \x_{t} gives \mathpzc{x}_{\mkern4mu t}

% Define a special case for \bar\x
\def\bx{\bar{\mathpzc{x}}}
\def\bxsub_#1{\bar{\mathpzc{x}}_{\mkern4mu #1}}
\def\barx{\@ifnextchar_{\bxsub}{\bx}}

% Define a special case for \check\x
\def\cx{\check{\mathpzc{x}}}
\def\cxsub_#1{\check{\mathpzc{x}}_{\mkern4mu #1}}
\def\checkx{\@ifnextchar_{\cxsub}{\bx}}

% Define a special case for \hat\x
\def\hx{\hat{\mathpzc{x}}}
\def\hxsub_#1{\hat{\mathpzc{x}}_{\mkern4mu #1}}
\def\hatx{\@ifnextchar_{\hxsub}{\bx}}

% Define a new epsilon for
\def\mveps{\mathcal{\varepsilon}}

\makeatother

% This make the figure & table centering automatically
\makeatletter
\g@addto@macro\@floatboxreset\centering
\makeatother

\title{Error State Kalman Filter on Matrix Lie Groups}
\author{Yi-Chen Zhang}
\date{\today}

\begin{document}

\maketitle

\tableofcontents
% This article summarizes the results from \citet{barrau_invariant_2017,
% barrau_invariant_2018}, \citet{arsenault_practical_2019}, and
% \citet{sola_micro_2021}. Additionally, the notations used in these papers have
% been unified in this work.

\section{Introduction}
\addcontentsline{toc}{subsection}{[put outline, related work, scope, and
contribute here]}

\section{Mathematical Prelimiaries}
\subsection{Lie Groups Overview}
\begin{definition}
  \textit{(Right-operators)} Let $\mathcal{X}$, $\mathcal{Y}\in \mathcal{M}$ The
  be elements in Lie group and $\prescript{\x}{}{\tau}\in
  \mathcal{T}_{\x}\mathcal{M}$ be a local tangent vector. We define the
  operator-pair:
  \begin{align*}
    \text{right-}\oplus: & \hspace{1ex}\mathcal{Y}=\mathcal{X}\oplus
    \prescript{\x}{}{\bm{\tau}}=\mathcal{X}\circ\Exp(\prescript{\x}{}{\bm{\tau}})\\
    \text{right-}\ominus: & \hspace{1ex} \prescript{\x}{}{\bm{\tau}} =
    \mathcal{Y}\ominus\mathcal{X}=\Log(\mathcal{X}^{-1}\circ\mathcal{Y})
  \end{align*}
\end{definition}
The above definition use a local tangent vector for ``addition'' and
``subtraction''. We call this right-$\oplus$ and right-$\ominus$ because
$\bm{\tau}$ appears in the right hand side expression of the composition. Since
the matrix operation is generally non-commutable, we further define the left
operators for the right counter-part. 
\begin{definition}
  \textit{(Left-operators)} Let $\mathcal{X}$, $\mathcal{Y}\in \mathcal{M}$ The
  be elements in Lie group and $\prescript{\mveps}{}{\tau}\in
  \mathcal{T}_{\mveps}\mathcal{M}$ be a global tangent vector. We define the
  operator-pair:
  \begin{align*}
    \text{left-}\oplus: & \hspace{1ex}\mathcal{Y}=
    \prescript{\mveps}{}{\bm{\tau}}\oplus\mathcal{X}=
    \Exp(\prescript{\mveps}{}{\bm{\tau}})\circ\mathcal{X}\\
    \text{left-}\ominus: & \hspace{1ex} \prescript{\mveps}{}{\bm{\tau}} =
    \mathcal{Y}\ominus\mathcal{X}=\Log(\mathcal{Y}\circ\mathcal{X}^{-1})
  \end{align*}
\end{definition}

% The other approach is from \citet{sola_micro_2021}, where they defines the error
% on the tangent space and applying the error-state extended Kalman filter on the
% Lie Group. To facilitate this, they introduce the concepts of right-plus and
% right-minus operators as follows:
\subsection{Jacobians on Lie Groups}
[put useful Jacobian result from \citet{sola_micro_2021} here]

\section{Error State Kalman Filter on Matrix Lie Groups}
\subsection{Error State Kalman Filtering}
Throughout this section, we use $\check{\x}$ to denote the predicted state and 
$\hat{\x}$ to denote the corrected state. The discrete-time prediction step is
given by:
\begin{align}
  \checkx_{t} & = f(\hatx_{t-1},\bm{u}_{t})\\
  \check{\bm{\Sigma}}_{t} & = \bm{F}_{t}\hat{\bm{\Sigma}}_{t-1}\bm{F}_{t}^{T}
  + \bm{W}_{t}\bm{Q}_{t}\bm{W}_{t}^{T}
\end{align}
The discrete-time Jacobians are derived in Sections
\ref{sec:left-disc-time-pred} and \ref{sec:right-disc-time-pred} for left- and
right-invariant errors, respectively.

The correction step is performed when a measurement $\bm{y}_{t}$ becomes
available. The left- and right-invariant measurement models are:
\begin{align}
  \label{eq:left_invariant_measurement}
  \bm{y}_{t} &= \x_{t}\cdot \bm{b}+\bm{\delta}_{t}\\
  \label{eq:right_invariant_measurement}
  \bm{y}_{t} &= \x_{t}^{-1}\cdot \bm{b}+\bm{\delta}_{t},
\end{align}
where $\bm{b}$ is a known vector and $\bm{\delta}_{t}\sim N(\bm{0},\bm{R}_{t})$.
Typically, a first-order Taylor series expansion of the measurement model is
used to obtain the measurement model and noise Jacobians, $\bm{H}_{t}$ and
$\bm{V}_{t}$. Here, we still apply the Taylor expansion, the difference is we
are expanding on the Matrix Lie groups. The case for right- and left-form
measurement models are covered in Seciton \ref{sec:left-disc-time-corr} and
\ref{sec:right-disc-time-corr}, respectively.

The Kalman gain at time $t$ is computed using
\begin{equation}
  \bm{K}_{t}=\check{\bm{\Sigma}}_{t}\bm{H}_{t}^{T}\left(\bm{H}_{t}
    \check{\bm{\Sigma}}_{t}\bm{H}_{t}^{T}+
    \bm{V}_{t}\bm{R}_{t}\bm{V}_{t}^{T}\right)^{-1}
\end{equation}

The covariance update is
\begin{equation}
  \hat{\bm{\Sigma}}_{t}=(\mathpzc{I}-\bm{K}_{t}\bm{H}_{t})\check{\bm{\Sigma}}_{t}
  (\mathpzc{I}-\bm{K}_{t}\bm{H}_{t})^{T}+
  \bm{K}_{t}\bm{V}_{t}\bm{R}_{t}\bm{V}_{t}^{T}\bm{K}_{t}^{T}
\end{equation}

\subsection{Right-form}
The right-form of ESKF on matrix Lie group corresponds to the left-invariant
EKF (LIEKF). For a left-invariant measurement model of the form
\eqref{eq:left_invariant_measurement}, we have derived the prediction and
correction steps.

\subsubsection{Prediction}
\label{sec:left-disc-time-pred} 
Using Taylor expansion of $\x_{t-1}$ around $\barx_{t-1}$ and
$\bm{\epsilon}_{t}$ around $\bm{0}$, we have:
\begin{align*}
  \x_{t} & = f(\x_{t-1},\bm{u}_{t}, \bm{\epsilon})\\
  & \approx f(\barx_{t-1},\bm{u}_{t},\bm{0})\oplus
  \left.\frac{\prescript{\x}{}{\partial}f}{\partial \x}\right|_{\x_{t-1}
  = \barx_{t-1}}(\x_{t-1}\ominus\barx_{t-1})\oplus
  \left.\frac{\prescript{\x}{}{\partial}f}
    {\partial\bm{\epsilon}}\right|_{\bm{\epsilon}_{t}=
  \bm{0}}(\bm{\epsilon}_{t}-\bm{0})\\
  & = \barx_{t}\oplus\prescript{\x}{}{\bm{F}_{t}}(\x_{t-1}\ominus\barx_{t-1})
  \oplus\prescript{\x}{}{\bm{W}_{t}}\bm{\epsilon}_{t}
\end{align*}

Let $\prescript{\x}{}{\bm{\xi}}=\x\ominus\barx=\Log(\barx^{-1}\circ\x).$ From
this definition, we have $\x=\barx\Exp(\prescript{\x}{}{\bm{\xi}})$. We can now
rewrite the earlier equation as:
\[
  \prescript{\x}{}{\bm{\xi}_{t}}=\prescript{\x}{}{\bm{F}_{t}}
  \prescript{\x}{}{\bm{\xi}_{t-1}}+\prescript{\x}{}{\bm{W}_{t}}\bm{\epsilon}_{t}
\]
where the Jacobians are defined as follows:
\begin{align*}
  \prescript{\x}{}{\bm{F}}_{t} & = 
  \left.\frac{\prescript{\x}{}{\partial}f}{\partial \x}\right|_{\x_{t-1}
  =\barx_{t-1}}\\
  & = \lim_{\x_{t-1}\rightarrow\barx_{t-1}}
  \frac{f(\x_{t-1},\bm{u}_{t},\bm{0})\ominus f(\barx_{t-1},\bm{u}_{t},\bm{0})}
  {\x_{t-1}\ominus\barx_{t-1}}\\
  & = \lim_{\prescript{\x}{}{\bm{\xi}_{t-1}}\rightarrow \bm{0}}
  \frac{f(\x_{t-1}\oplus\prescript{\x}{}{\bm{\xi}_{t-1}},\bm{u}_{t},\bm{0})
  \ominus f(\x_{t-1},\bm{u}_{t},\bm{0})}{\prescript{\x}{}{\bm{\xi}_{t-1}}}\\
  & = \frac{\prescript{\x}{}{D}f(\x_{t-1},\bm{u}_{t},\bm{0})}{D\x_{t-1}}\\
  & = \prescript{\x}{}{\bm{J}}_{\x_{t-1}}^{f(\x_{t-1},\bm{u}_{t},\bm{0})}
\end{align*}
and
\begin{align*}
  \prescript{\x}{}{\bm{W}}_{t} & =
  \left.\frac{\prescript{\x}{}{\partial} f}{\partial \bm{\epsilon}}
    \right|_{\bm{\epsilon}_{t}=\bm{0}}\\
  & = \lim_{\bm{\epsilon}_{t}\rightarrow \bm{0}}
  \frac{f(\barx_{t-1},\bm{u}_{t},\bm{\epsilon}_{t})\ominus
  f(\barx_{t-1},\bm{u}_{t},\bm{0})}{\bm{\epsilon}_{t}}\\
  & = \prescript{\x}{}{\bm{J}}^{f(\barx_{t-1},\bm{u}_{t},\bm{\epsilon}_{t})}_{\bm{\epsilon}_{t}}
\end{align*}

\subsubsection{Correction}
\label{sec:left-disc-time-corr} 
Define the innovation $\bm{z}_{t}$ such that
\begin{align*}
  \bm{z}_{t} & = \barx_{t}^{-1}\cdot (\bm{y}_{t}-\bar{\bm{y}}_{t})\\
  & = \barx_{t}^{-1}\cdot \bm{y}_{t}-\bm{b}
\end{align*}
Now let $h(\x_{t})=(\barx_{t}^{-1}\x_{t})\cdot \bm{b}$ and note that
\begin{align*}
  \barx_{t}^{-1} \cdot \bm{y}_{t}
  & = \barx_{t}^{-1}(\x_{t} \cdot \bm{b}+\bm{\delta}_{t})\\
  & = \barx_{t}^{-1}\x_{t} \cdot \bm{b}+\barx_{t}^{-1}\cdot \bm{\delta}_{t}\\
  & = h(\barx_{t})
  +\left.\frac{\partial h}{\partial\x}
    \right|_{\x_{t}=\barx_{t}}(\x_{t}\ominus\barx_{t})
    +\barx_{t}^{-1}\cdot\bm{\delta}_{t}\\
  & = \bm{b} + \prescript{\x}{}{\bm{H}}_{t}\prescript{\x}{}{\bm{\xi}}_{t}
    +\barx_{t}^{-1}\cdot
  \bm{\delta}_{t}\\
\end{align*}
So, we can rewrite the innovation as
\[
  \bm{z}_{t} = \prescript{\x}{}{\bm{H}}_{t}{}^{\x}\bm{\xi}_{t}+\bm{V}_{t}\bm{\delta}_{t}\\
\]
where
\[
  \prescript{\x}{}{\bm{H}}_{t}=\left.\frac{\prescript{\x}{}{\partial}h}
    {\partial\x}\right|_{\x_{t}=\barx_{t}}\\
\]

\subsection{Left-form}
The left-form of ESKF on matrix Lie group corresponds to the right-invariant EKF
(RIEKF). For a right-invariant measurement model of the form
\eqref{eq:right_invariant_measurement}, we have derived the prediction and
correction steps.
\subsubsection{Prediction}
\label{sec:right-disc-time-pred} 
Next, using Taylor expansion of $\x_{t-1}$ around $\barx_{t-1}$ and
$\bm{\epsilon}_{t}$ around $\bm{0}$, we have:
\begin{align*}
  \x_{t} & = f(\x_{t-1},\bm{u}_{t}, \bm{\epsilon})\\
  & \approx \left.\frac{\prescript{\mveps}{}{\partial}f}
    {\partial\bm{\epsilon}}\right|_{\bm{\epsilon}_{t}=
  \bm{0}}(\bm{\epsilon}_{t}-\bm{0}) \oplus
  \left.\frac{\prescript{\mveps}{}{\partial}f}{\partial \x}\right|_{\x_{t-1}
  = \barx_{t-1}}(\x_{t-1}\ominus\barx_{t-1})\oplus
  f(\barx_{t-1},\bm{u}_{t},\bm{0})\\
  & = \prescript{\mveps}{}{\bm{W}_{t}}\bm{\epsilon}_{t}\oplus
  \prescript{\mveps}{}{\bm{F}_{t}}(\x_{t-1}\ominus\barx_{t-1})\oplus\barx_{t}
\end{align*}

Let $\prescript{\mveps}{}{\bm{\xi}}=\x\ominus\barx=\Log(\x\circ\barx^{-1}).$ From
this definition, we have $\x=\Exp(\prescript{\mveps}{}{\bm{\xi}})\barx$. We can
now rewrite the earlier equation as:
\[
  \prescript{\mveps}{}{\bm{\xi}_{t}}=\prescript{\mveps}{}{\bm{F}_{t}}
  \prescript{\mveps}{}{\bm{\xi}_{t-1}}+
  \prescript{\mveps}{}{\bm{W}_{t}}\bm{\epsilon}_{t}
\]
where the Jacobians are defined as follows:
\begin{align*}
  \prescript{\mveps}{}{\bm{F}}_{t} & = 
  \left.\frac{\prescript{\mveps}{}{\partial}f}{\partial \x}\right|_{\x_{t-1}
  =\barx_{t-1}}\\
  & = \lim_{\x_{t-1}\rightarrow\barx_{t-1}}
  \frac{f(\x_{t-1},\bm{u}_{t},\bm{0})\ominus f(\barx_{t-1},\bm{u}_{t},\bm{0})}
  {\x_{t-1}\ominus\barx_{t-1}}\\
  & = \lim_{\prescript{\mveps}{}{\bm{\xi}_{t-1}}\rightarrow \bm{0}}
  \frac{f(\prescript{\mveps}{}{\bm{\xi}_{t-1}\oplus\x_{t-1}},\bm{u}_{t},\bm{0})
  \ominus f(\x_{t-1},\bm{u}_{t},\bm{0})}{\prescript{\mveps}{}{\bm{\xi}_{t-1}}}\\
  & = \frac{\prescript{\mveps}{}{D}f(\x_{t-1},\bm{u}_{t},\bm{0})}{D\x_{t-1}}\\
  & = \textbf{Ad}_{f(\x_{t-1},\bm{u}_{t},\bm{0})}
  \prescript{\x}{}{\bm{J}}^{f(\x_{t-1},\bm{u}_{t},\bm{0})}_{\x_{t-1}}
  \textbf{Ad}_{\x_{t-1}}^{-1}
  %& = \prescript{\mveps}{}{\bm{J}}_{\x_{t-1}}^{f(\x_{t-1},\bm{u}_{t},\bm{0})}\\
  %& = 
\end{align*}
and
\begin{align*}
  \prescript{\mveps}{}{\bm{W}}_{t} & =
  \left.\frac{\prescript{\mveps}{}{\partial} f}{\partial \bm{\epsilon}}
    \right|_{\bm{\epsilon}_{t}=\bm{0}}\\
  & = \lim_{\bm{\epsilon}_{t}\rightarrow \bm{0}}
  \frac{f(\barx_{t-1},\bm{u}_{t},\bm{\epsilon}_{t})\ominus
  f(\barx_{t-1},\bm{u}_{t},\bm{0})}{\bm{\epsilon}_{t}}\\
  & = \prescript{\mveps}{}{\bm{J}}^{f(\barx_{t-1},\bm{u}_{t},\bm{\epsilon}_{t})}_{\bm{\epsilon}_{t}}
\end{align*}

\subsubsection{Correction}
\label{sec:right-disc-time-corr} 
Define the innovation $z_{t}$ such that
\begin{align*}
  \bm{z}_{t} & = \barx_{t} \cdot (\bm{y}_{t}-\bar{\bm{y}}_{t})\\
  & = \barx_{t} \cdot \bm{y}_{t}-\bm{b}
\end{align*}
Now let $h(\x_{t})=(\barx_{t}\x_{t}^{-1}) \cdot \bm{b}$ and note that
\begin{align*}
  \barx_{t} \cdot \bm{y}_{t}
  & = \barx_{t}(\x_{t}^{-1} \cdot \bm{b}+\bm{\delta}_{t})\\
  & = \barx_{t}\x_{t}^{-1} \cdot \bm{b}+\barx_{t}\cdot \bm{\delta}_{t}\\
  & = h(\barx_{t})+\left.\frac{\prescript{\mveps}{}{\partial}h}{\partial\x}
    \right|_{\x_{t}=\barx_{t}}(\x_{t}\ominus\barx_{t})
    +\barx_{t}\cdot\bm{\delta}_{t}\\
    & = \bm{b} + \prescript{\mveps}{}{\bm{H}}_{t}\prescript{\mveps}{}{\bm{\xi}}_{t}+\barx_{t}\cdot
  \bm{\delta}_{t}\\
\end{align*}
So, we can rewrite the innovation as
\[
  \bm{z}_{t} = \prescript{\mveps}{}{\bm{H}}_{t}{}^{\x}\bm{\xi}_{t}+\bm{V}_{t}\bm{\delta}_{t}\\
\]
where
\[
  \prescript{\mveps}{}{\bm{H}}_{t}=\left.\frac{\prescript{\mveps}{}{\partial}h}{\partial\x}\right|_{\x_{t}=\barx_{t}}\\
\]

\subsection{Mixed Left and Right}
One can switch beteen the left and right error forms through the use of the
adjoint map.

\begin{align*}
  \prescript{\x}{}{\bm{\xi}_{t}}
  & = \Log\left(\barx_{t}\circ\x_{t}^{-1}\right)\\
  & = \Log\left(\barx_{t}\Exp(\prescript{\mveps}{}{\bm{\xi}_{t}})\barx_{t}^{-1}\right)\\
  & = \Log\left(\exp(\barx_{t}\prescript{\mveps}{}{\bm{\xi}_{t}^{\wedge}}\barx_{t}^{-1})\right)\\
  & = \Log\left(\Exp(\textbf{Ad}_{\barx_{t}}\prescript{\mveps}{}{\bm{\xi}_{t}})\right)\\
  & = \textbf{Ad}_{\barx_{t}}\prescript{\mveps}{}{\bm{\xi}_{t}}
\end{align*}
This transformation is exact, which means that we can easily switch between the
covariance of the left and right invariant errors using
\begin{equation}
  \prescript{\x}{}{\bm{\Sigma}_{t}} = \textbf{Ad}_{\barx_{t}}
  \prescript{\mveps}{}{\bm{\Sigma}_{t}}\textbf{Ad}_{\barx_{t}}^{T}
\end{equation}

\section{Simulation Experiment}
\addcontentsline{toc}{subsection}{[Compare with other Kalman Filter families]}

% \subsection{Example: $SE(2)$}
% \subsubsection{System Modelling}
% We consider the robot pose $\x$ in $SE(2)$ and the GPS measurement $y$ is in
% $\mathbb{R}^{2}$
% \[
%   \x=\begin{bmatrix}
%     \bm{R} & \bm{t}\\
%     \bm{0}^{T} & 1
%   \end{bmatrix}\in\text{SE}(2)
% \]
% The control signal $\bm{u}$ is in $\mathfrak{se}(2)$ and is corrupted by
% additive Gaussian noise $\bm{\epsilon}$, which has a mean of $\bm{0}$ and a
% covariance $\bm{Q}$. Upon receiving a control $\bm{u}$, the robot pose is
% updated as follows:
% \[
% \x_{t}% & = f(\x_{t-1}, \bm{u}_{t},\bm{\epsilon}_{t})\\
% = \x_{t-1}\Exp(\bm{u}_{t}+\bm{\epsilon}_{t}).
% \]

% \subsubsection{Right-form Derivation}
% In order to derive a LIEKF, the exteroceptive measurement model most be
% left-invariant. In this case, appropriate measurements are position
% measurements, which could be from a GPS receiver.

% The GPS measurement $\bm{y}$ is in Cartesian form for simplicity, and the noise
% is denoted $\bm{\delta}$, which has a mean of $\bm{0}$ and a covariance of
% $\bm{R}$.
% \[
% \bm{y}_{t} %& = h(\chi_{t}, b) + \mathbf{\delta}_{t}\\
% = \x_{t} \cdot \bm{b}+\bm{\delta}_{t}
% \]
% where $\bm{b}=\bm{0}$ for rigid motion action in this case.

% For predict step:

% For $\prescript{\x}{}{\bm{F}}_{t}$:
% \begin{align*}
%   \prescript{\x}{}{\bm{F}}_{t} & = 
%   \prescript{\x}{}{\bm{J}}_{\x_{t-1}}^{f(\x_{t-1},\bm{u}_{t},\bm{0})}\\
%   & = \prescript{\x}{}{\bm{J}}_{\x_{t-1}}^{\x_{t-1}\oplus\bm{u}_{t}}\\
%   & = \prescript{\x}{}{\bm{J}}_{\x_{t-1}}^{\x_{t-1}\Exp(\bm{u}_{t})}\\
%   & = \text{Ad}_{\Exp(\bm{u}_{t})}^{-1}
% \end{align*}

% For $\prescript{\x}{}{\bm{W}}_{t}$:
% \begin{align*}
%   \prescript{\x}{}{\bm{W}}_{t} & =
%   \prescript{\x}{}{\bm{J}}^{f(\barx_{t-1},\bm{u}_{t},
%   \bm{\epsilon}_{t})}_{\bm{\epsilon}_{t}}\\
%   & = \prescript{\x}{}{\bm{J}}^{(\barx_{t-1}\oplus\bm{u}_{t})
%   \oplus\bm{\epsilon}_{t}}_{\bm{\epsilon}_{t}}\\
%   & = \prescript{\x}{}{\bm{J}}_{r}(\bm{\epsilon}_{t})\\
%   & \approx \mathpzc{I}
% \end{align*}

% Therefore, the predicted state and error covariance are
% \begin{align*}
%   \checkx_{t} & = \barx_{t}=\barx_{t-1}\oplus \bm{u}_{t}\\
%   \check{\bm{\Sigma}}_{t} & = \bm{F}_{t}\hat{\bm{\Sigma}}_{t-1}\bm{F}_{t}^{T}
%   +\bm{W}_{t}\bm{Q}_{t}\bm{W}_{t}^{T}
% \end{align*}

% For update step:
% \begin{align*}
%   \bm{y}_{t} & = h(\x_{t}, \bm{b})+\bm{\delta}_{t},\hspace{1ex}\text{with~}
%   \bm{b}=\bm{0}\\
%   & = \x_{t}\cdot \bm{b}+ \bm{\delta}_{t}
% \end{align*}

% Now, define the innovation $z_{t}$ such that
% \begin{align*}
%   \bm{z}_{t} & = \barx_{t}^{-1}\cdot\left(\bm{y}_{t}-\bm{\bar{y}}_{t}\right)\\
%   & = \barx_{t}^{-1}\x_{t}\cdot\bm{b}+\barx_{t}^{-1}\cdot\bm{\delta}_{t}-\bm{b}
% \end{align*}

% \begin{align*}
%   \bm{H}_{t} & =
%   \left.\frac{\partial h}{\partial\x}\right|_{\x_{t}=\barx_{t}}\\
%   & = \lim_{\x_{t} \rightarrow \barx_{t}}\frac{h(\x_{t})-h(\barx_{t})}
%   {\x_{t} \ominus \barx_{t}}\\
%   & = \lim_{\prescript{\x}{}{\bm{\xi}_{t}}\rightarrow \bm{0}}
%   \frac{\Exp(\prescript{\x}{}{\bm{\xi}_{t}})\cdot\bm{b}-\bm{b}}
%   {\prescript{\x}{}{\bm{\xi}_{t}}}\\
%   & = \lim_{\prescript{\x}{}{\bm{\xi}_{t}}\rightarrow \bm{0}}
%   \frac{(\mathpzc{I}+\prescript{\x}{}{\bm{\xi}}_{t}^{\wedge})\cdot \bm{b}-\bm{b}}
%   {\prescript{\x}{}{\bm{\xi}_{t}}}\\
%   & = \lim_{\prescript{\x}{}{\bm{\xi}_{t}}\rightarrow\bm{0}}
%   \frac{\phi^{\wedge}\bm{b}+\bm{\rho}}{\prescript{\x}{}{\bm{\xi}_{t}}}\\
%   & = \begin{bmatrix}
%     \left.\frac{\partial\phi^{\wedge}\bm{b}+\bm{\rho}}
%       {\partial\bm{\rho}}\right|_{\bm{\rho}=\bm{0}} &
%     \left.\frac{\partial\phi^{\wedge}\bm{b}+\bm{\rho}}
%       {\partial\phi}\right|_{\phi=0}\end{bmatrix}\\
%   & = \begin{bmatrix}
%     \mathpzc{I} & \bm{0}
%   \end{bmatrix}
% \end{align*}
% and $\bm{V}_{t}=\bar{\bm{R}}_{t}^{T}$.

% Furthermore, we have $\Exp(\prescript{\x}{}{\bm{\xi}_{t}})=
% \barx_{t}^{-1}\x_{t}$, which implies that
% $\x_{t}=\barx_{t}\Exp(\prescript{\x}{}{\bm{\xi}_{t}})$. This suggest the update 
% of $\x_{t}$ is
% \[
%   \hatx_{t}=\checkx_{t}\Exp(\bm{K}_{t}\bm{z}_{t}),
% \]
% where $\bm{K}_{t}=\check{\bm{\Sigma}}_{t}\bm{H}_{t}\bm{S}_{t}^{-1}$ and
% $\bm{S}_{t}=\bm{H}_{t}\check{\bm{\Sigma}}_{t}\bm{H}_{t}^{T}
% +\bm{V}_{t}\bm{R}_{t}\bm{V}_{t}^{T}$.

% Thus, the updated state and error covariance are given by:
% \begin{align}
%   \hatx_{t} & = \checkx_{t}\Exp(\bm{K}_{t}\bm{z}_{t})\\
%   \hat{\bm{\Sigma}}_{t} & =
%   (\mathpzc{I}-\bm{K}_{t}\bm{H}_{t})\check{\bm{\Sigma}}_{t}
% \end{align}

% \subsubsection{Left-form Derivation}
% To derive a RIEKF, the exteroceptive measurement model must be right-invariant.
% Suitable measurements for this case include landmark observations in the body
% frame, which can be obtained from sensors such as LiDAR or a camera.

% The landmark measurements are of the range and bearing type, though they are put
% in Cartesian form for simplicity. The noise $\bm{\delta}$ is zero mean Gaussian,
% and is specified with a covariance matrix $\bm{R}$. We notice the rigid motion
% action 
% \[
%   \bm{y}_{t} = h(\x, \bm{b}) = \x_{t}^{-1}\cdot \bm{b}+\bm{\delta}
% \]
% where $\bm{b}\in\mathbb{R}^{2}$ is the landmark position.

% For predict step:

% For $\prescript{\mveps}{}{\bm{F}}_{t}$:
% \begin{align*}
%   \prescript{\mveps}{}{\bm{F}}_{t} & =
%   \textbf{Ad}_{f(\x_{t-1},\bm{u}_{t},\bm{0})}
%   \prescript{\x}{}{\bm{J}}^{f(\x_{t-1},\bm{u}_{t},\bm{0})}_{\x_{t-1}}
%   \textbf{Ad}_{\x_{t-1}}^{-1}\\
%   & = \textbf{Ad}_{\x_{t-1}\Exp(\bm{u}_{t})}
%   \textbf{Ad}^{-1}_{\Exp(\bm{u}_{t})}\textbf{Ad}_{\x_{t-1}}^{-1}\\
%   & = \textbf{Ad}_{\x_{t-1}\Exp(\bm{u}_{t})}
%   \textbf{Ad}^{-1}_{\x_{t-1}\Exp(\bm{u}_{t})}\\
%   & = \mathpzc{I}
% \end{align*}

% \begin{align*}
%   \prescript{\mveps}{}{\bm{W}}_{t} & =
%   \left.\frac{\prescript{\mveps}{}{\partial} f}{\partial \bm{\epsilon}}
%     \right|_{\bm{\epsilon}_{t}=\bm{0}}\\
%   & = \lim_{\bm{\epsilon}_{t}\rightarrow \bm{0}}
%   \frac{f(\barx_{t-1},\bm{u}_{t},\bm{\epsilon}_{t})\ominus
%   f(\barx_{t-1},\bm{u}_{t},\bm{0})}{\bm{\epsilon}_{t}}\\
%   & = \lim_{\bm{\epsilon}_{t}\rightarrow \bm{0}}
%   \frac{\Log\left(\barx_{t-1}\Exp(\bm{u}_{t})\Exp(\bm{\epsilon}_{t})
%   \left[\barx_{t-1}\Exp(\bm{u}_{t})\right]^{-1}\right)}{\bm{\epsilon}_{t}}\\
%   & = \lim_{\bm{\epsilon}_{t}\rightarrow \bm{0}}
%   \frac{\Log\left(\barx_{t}\Exp(\bm{\epsilon}_{t})\barx_{t}^{-1}\right)}{\bm{\epsilon}_{t}}\\
%   & = \lim_{\bm{\epsilon}_{t}\rightarrow \bm{0}}
%   \frac{\Log\left(\barx_{t}\exp(\bm{\epsilon}_{t}^{\wedge})\barx_{t}^{-1}\right)}{\bm{\epsilon}_{t}}\\
%   & = \lim_{\bm{\epsilon}_{t}\rightarrow \bm{0}}
%   \frac{\Log\left(\exp(\barx_{t}\bm{\epsilon}_{t}^{\wedge}\barx_{t}^{-1})\right)}{\bm{\epsilon}_{t}}\\
%   & = \lim_{\bm{\epsilon}_{t}\rightarrow \bm{0}}
%   \frac{\Log\left(\Exp(\textbf{Ad}_{\barx_{t}}\bm{\epsilon}_{t})\right)}{\bm{\epsilon}_{t}}\\
%   & = \lim_{\bm{\epsilon}_{t}\rightarrow \bm{0}}
%   \frac{\textbf{Ad}_{\barx_{t}}\bm{\epsilon}_{t}}{\bm{\epsilon}_{t}}\\
%   & = \frac{\partial\textbf{Ad}_{\barx_{t}}\bm{\epsilon}_{t}}{\partial\bm{\epsilon}_{t}}\\
%   & = \textbf{Ad}_{\barx_{t}}
% \end{align*}

% Therefore, the predicted state and error covariance are
% \begin{align*}
%   \checkx_{t} & = \barx_{t}=\barx_{t-1}\oplus \bm{u}_{t}\\
%   \check{\bm{\Sigma}}_{t} & = \bm{F}_{t}\hat{\bm{\Sigma}}_{t-1}\bm{F}_{t}^{T}
%   +\bm{W}_{t}\bm{Q}_{t}\bm{W}_{t}^{T}
% \end{align*}

% For update step:
% \begin{align*}
%   \bm{y}_{t} & = h(\x_{t}, \bm{b})+\bm{\delta}_{t}\\
%   & = \x_{t}^{-1} \cdot \bm{b}+ \bm{\delta}_{t}
% \end{align*}

% Now, define the innovation $z_{t}$ such that
% \begin{align*}
%   \bm{z}_{t} & = \barx_{t}\cdot\left(\bm{y}_{t}-\bm{\bar{y}}_{t}\right)\\
%   & = \barx_{t}\x_{t}^{-1}\cdot\bm{b}+\barx_{t}\cdot\bm{\delta}_{t}-\bm{b}
% \end{align*}

% \begin{align*}
%   \bm{H}_{t} & =
%   \left.\frac{\partial h}{\partial\x}\right|_{\x_{t}=\barx_{t}}\\
%   & = \lim_{\x_{t} \rightarrow \barx_{t}}\frac{h(\x_{t})-h(\barx_{t})}
%   {\x_{t} \ominus \barx_{t}}\\
%   & = \lim_{\prescript{\mveps}{}{\bm{\xi}_{t}}\rightarrow \bm{0}}
%   \frac{\Exp(-\prescript{\mveps}{}{\bm{\xi}_{t}})\cdot\bm{b}-\bm{b}}
%   {\prescript{\x}{}{\bm{\xi}_{t}}}\\
%   & = \lim_{\prescript{\mveps}{}{\bm{\xi}_{t}}\rightarrow \bm{0}}
%   \frac{(\mathpzc{I}-\prescript{\mveps}{}{\bm{\xi}}_{t}^{\wedge})\cdot\bm{b}-\bm{b}}
%   {\prescript{\mveps}{}{\bm{\xi}_{t}}}\\
%   & = \lim_{\prescript{\mveps}{}{\bm{\xi}_{t}}\rightarrow\bm{0}}
%   \frac{-\phi^{\wedge}\bm{b}-\bm{\rho}}{\prescript{\x}{}{\bm{\xi}_{t}}}\\
%   & = \begin{bmatrix}
%     \left.\frac{\partial(-\phi^{\wedge}\bm{b}-\bm{\rho})}
%       {\partial\bm{\rho}}\right|_{\bm{\rho}=\bm{0}} &
%     \left.\frac{\partial(-\phi^{\wedge}\bm{b}-\bm{\rho})}
%       {\partial\phi}\right|_{\phi=0}\end{bmatrix}\\
%   & = \begin{bmatrix}
%     -\mathpzc{I} & -[1]_{\times}\bm{b}
%   \end{bmatrix}
% \end{align*}
% and $\bm{V}_{t}=\bar{\bm{R}}_{t}^{T}$.

% Furthermore, we have $\Exp(\prescript{\mveps}{}{\bm{\xi}_{t}})=
% \x_{t}\barx_{t}^{-1}$, which implies that
% $\x_{t}=\Exp(\prescript{\mveps}{}{\bm{\xi}_{t}})\barx_{t}$. This suggest the update 
% of $\x_{t}$ is
% \[
%   \hatx_{t}=\Exp(\bm{K}_{t}\bm{z}_{t})\checkx_{t},
% \]
% where $\bm{K}_{t}=\check{\bm{\Sigma}}_{t}\bm{H}_{t}\bm{S}_{t}^{-1}$ and
% $\bm{S}_{t}=\bm{H}_{t}\check{\bm{\Sigma}}_{t}\bm{H}_{t}^{T}
% +\bm{V}_{t}\bm{R}_{t}\bm{V}_{t}^{T}$.

% Thus, the updated state and error covariance are given by:
% \begin{align*}
%   \hatx_{t} & = \Exp(\bm{K}_{t}\bm{z}_{t})\checkx_{t}\\
%   \hat{\bm{\Sigma}}_{t} & =
%   (\mathpzc{I}-\bm{K}_{t}\bm{H}_{t})\check{\bm{\Sigma}}_{t}
% \end{align*}

\subsection{Example: $SE(3)$}
\subsubsection{System Modelling}
We consider the robot pose $\x$ in $SE(3)$ and the GPS measurement $y$ is in
$\mathbb{R}^{3}$
\[
  \x=\begin{bmatrix}
    \bm{R} & \bm{t}\\
    \bm{0}^{T} & 1
  \end{bmatrix}\in\text{SE}(3)
\]
The control signal $\bm{u}$ is in $\mathfrak{se}(3)$:
\[
  \bm{u}=\begin{bmatrix}
    \bm{\nu}\\
    \bm{\omega}
  \end{bmatrix}\in\mathfrak{se}(3)
\]
where $\bm{\nu}$ is the linear velocity and $\bm{\omega}$ is the angular
velocity. We assume the control $\bm{u}$ is corrupted by additive Gaussian
noise $\bm{\epsilon}$, which has a mean of $\bm{0}$ and a covariance $\bm{Q}$.
Upon receiving a control $\bm{u}$, the robot pose is updated as follows:
\[
\x_{t}% & = f(\x_{t-1}, \bm{u}_{t},\bm{\epsilon}_{t})\\
= \x_{t-1}\Exp(\bm{u}_{t}+\bm{\epsilon}_{t}).
\]

\subsubsection{Right-form Derivation}
In order to derive a LIEKF, the exteroceptive measurement model most be
left-invariant. In this case, appropriate measurements are position
measurements, which could be from a GPS receiver.

The GPS measurement $\bm{y}$ is in Cartesian form for simplicity, and the noise
is denoted $\bm{\delta}$, which has a mean of $\bm{0}$ and a covariance of
$\bm{R}$.
\[
\bm{y}_{t} %& = h(\chi_{t}, b) + \mathbf{\delta}_{t}\\
= \x_{t} \cdot \bm{b}+\bm{\delta}_{t}
\]
where $\bm{b}=\bm{0}$ for rigid motion action in this case.

For predict step:

For $\prescript{\x}{}{\bm{F}}_{t}$:
\begin{align*}
  \prescript{\x}{}{\bm{F}}_{t} & = 
  \prescript{\x}{}{\bm{J}}_{\x_{t-1}}^{f(\x_{t-1},\bm{u}_{t},\bm{0})}\\
  & = \prescript{\x}{}{\bm{J}}_{\x_{t-1}}^{\x_{t-1}\oplus\bm{u}_{t}}\\
  & = \prescript{\x}{}{\bm{J}}_{\x_{t-1}}^{\x_{t-1}\Exp(\bm{u}_{t})}\\
  & = \text{Ad}_{\Exp(\bm{u}_{t})}^{-1}
\end{align*}

For $\prescript{\x}{}{\bm{W}}_{t}$:
\begin{align*}
  \prescript{\x}{}{\bm{W}}_{t} & =
  \prescript{\x}{}{\bm{J}}^{f(\barx_{t-1},\bm{u}_{t},
  \bm{\epsilon}_{t})}_{\bm{\epsilon}_{t}}\\
  & = \prescript{\x}{}{\bm{J}}^{(\barx_{t-1}\oplus\bm{u}_{t})
  \oplus\bm{\epsilon}_{t}}_{\bm{\epsilon}_{t}}\\
  & = \prescript{\x}{}{\bm{J}}_{r}(\bm{\epsilon}_{t})\\
  & \approx \mathpzc{I}
\end{align*}

Therefore, the predicted state and error covariance are
\begin{align*}
  \checkx_{t} & = \barx_{t}=\barx_{t-1}\oplus \bm{u}_{t}\\
  \check{\bm{\Sigma}}_{t} & = \bm{F}_{t}\hat{\bm{\Sigma}}_{t-1}\bm{F}_{t}^{T}
  +\bm{W}_{t}\bm{Q}_{t}\bm{W}_{t}^{T}
\end{align*}

For update step:

Define the innovation $z_{t}$ such that
\begin{align*}
  \bm{z}_{t} & = \barx_{t}^{-1}\cdot\left(\bm{y}_{t}-\bm{\bar{y}}_{t}\right)\\
  & = \barx_{t}^{-1}\x_{t}\cdot\bm{b}+\barx_{t}^{-1}\cdot\bm{\delta}_{t}-\bm{b}
\end{align*}

\begin{align*}
  \bm{H}_{t} & =
  \left.\frac{\partial h}{\partial\x}\right|_{\x_{t}=\barx_{t}}\\
  & = \lim_{\x_{t} \rightarrow \barx_{t}}\frac{h(\x_{t})-h(\barx_{t})}
  {\x_{t} \ominus \barx_{t}}\\
  & = \lim_{\prescript{\x}{}{\bm{\xi}_{t}}\rightarrow \bm{0}}
  \frac{\Exp(\prescript{\x}{}{\bm{\xi}_{t}})\cdot\bm{b}-\bm{b}}
  {\prescript{\x}{}{\bm{\xi}_{t}}}\\
  & = \lim_{\prescript{\x}{}{\bm{\xi}_{t}}\rightarrow \bm{0}}
  \frac{(\mathpzc{I}+\prescript{\x}{}{\bm{\xi}}_{t}^{\wedge})\cdot \bm{b}-\bm{b}}
  {\prescript{\x}{}{\bm{\xi}_{t}}}\\
  & = \lim_{\prescript{\x}{}{\bm{\xi}_{t}}\rightarrow\bm{0}}
  \frac{\bm{\phi}^{\wedge}\bm{b}+\bm{\rho}}{\prescript{\x}{}{\bm{\xi}_{t}}}\\
  & = \lim_{\prescript{\x}{}{\bm{\xi}_{t}}\rightarrow\bm{0}}
  \frac{-\bm{b}^{\wedge}\bm{\phi}+\bm{\rho}}{\prescript{\x}{}{\bm{\xi}_{t}}}\\
  & = \begin{bmatrix}
    \left.\frac{\partial (-\bm{b}^{\wedge}\bm{\phi}+\bm{\rho})}
      {\partial\bm{\rho}}\right|_{\bm{\rho}=\bm{0}} &
    \left.\frac{\partial (-\bm{b}^{\wedge}\bm{\phi}+\bm{\rho})}
      {\partial\bm{\phi}}\right|_{\bm{\phi}=\bm{0}}\end{bmatrix}\\
  & = \begin{bmatrix}
    \mathpzc{I} & -[\bm{b}]_{\times}
  \end{bmatrix}\\
  & = \begin{bmatrix}
    \mathpzc{I} & \bm{0}
  \end{bmatrix}
\end{align*}
and $\bm{V}_{t}=\bar{\bm{R}}_{t}^{T}$.

Furthermore, we have $\Exp(\prescript{\x}{}{\bm{\xi}_{t}})=
\barx_{t}^{-1}\x_{t}$, which implies that
$\x_{t}=\barx_{t}\Exp(\prescript{\x}{}{\bm{\xi}_{t}})$. This suggest the update 
of $\x_{t}$ is
\[
  \hatx_{t}=\checkx_{t}\Exp(\bm{K}_{t}\bm{z}_{t}),
\]
where $\bm{K}_{t}=\check{\bm{\Sigma}}_{t}\bm{H}_{t}\bm{S}_{t}^{-1}$ and
$\bm{S}_{t}=\bm{H}_{t}\check{\bm{\Sigma}}_{t}\bm{H}_{t}^{T}
+\bm{V}_{t}\bm{R}_{t}\bm{V}_{t}^{T}$.

Thus, the updated state and error covariance are given by:
\begin{align}
  \hatx_{t} & = \checkx_{t}\Exp(\bm{K}_{t}\bm{z}_{t})\\
  \hat{\bm{\Sigma}}_{t} & =
  (\mathpzc{I}-\bm{K}_{t}\bm{H}_{t})\check{\bm{\Sigma}}_{t}
\end{align}

\subsubsection{Left-form Derivation}
To derive a RIEKF, the exteroceptive measurement model must be right-invariant.
Suitable measurements for this case include landmark observations in the body
frame, which can be obtained from sensors such as LiDAR or a camera.

The landmark measurements are of the range and bearing type, though they are put
in Cartesian form for simplicity. The noise $\bm{\delta}$ is zero mean Gaussian,
and is specified with a covariance matrix $\bm{R}$. We notice the rigid motion
action 
\[
  \bm{y}_{t} = \x_{t}^{-1}\cdot \bm{b}+\bm{\delta}
\]
where $\bm{b}\in\mathbb{R}^{3}$ is the landmark position.

For predict step:

For $\prescript{\mveps}{}{\bm{F}}_{t}$:
\begin{align*}
  \prescript{\mveps}{}{\bm{F}}_{t} & =
  \textbf{Ad}_{f(\x_{t-1},\bm{u}_{t},\bm{0})}
  \prescript{\x}{}{\bm{J}}^{f(\x_{t-1},\bm{u}_{t},\bm{0})}_{\x_{t-1}}
  \textbf{Ad}_{\x_{t-1}}^{-1}\\
  & = \textbf{Ad}_{\x_{t-1}\Exp(\bm{u}_{t})}
  \textbf{Ad}^{-1}_{\Exp(\bm{u}_{t})}\textbf{Ad}_{\x_{t-1}}^{-1}\\
  & = \textbf{Ad}_{\x_{t-1}\Exp(\bm{u}_{t})}
  \textbf{Ad}^{-1}_{\x_{t-1}\Exp(\bm{u}_{t})}\\
  & = \mathpzc{I}
\end{align*}

\begin{align*}
  \prescript{\mveps}{}{\bm{W}}_{t} & =
  \left.\frac{\prescript{\mveps}{}{\partial} f}{\partial \bm{\epsilon}}
    \right|_{\bm{\epsilon}_{t}=\bm{0}}\\
  & = \lim_{\bm{\epsilon}_{t}\rightarrow \bm{0}}
  \frac{f(\barx_{t-1},\bm{u}_{t},\bm{\epsilon}_{t})\ominus
  f(\barx_{t-1},\bm{u}_{t},\bm{0})}{\bm{\epsilon}_{t}}\\
  & = \lim_{\bm{\epsilon}_{t}\rightarrow \bm{0}}
  \frac{\Log\left(\barx_{t-1}\Exp(\bm{u}_{t})\Exp(\bm{\epsilon}_{t})
  \left[\barx_{t-1}\Exp(\bm{u}_{t})\right]^{-1}\right)}{\bm{\epsilon}_{t}}\\
  & = \lim_{\bm{\epsilon}_{t}\rightarrow \bm{0}}
  \frac{\Log\left(\barx_{t}\Exp(\bm{\epsilon}_{t})\barx_{t}^{-1}\right)}{\bm{\epsilon}_{t}}\\
  & = \lim_{\bm{\epsilon}_{t}\rightarrow \bm{0}}
  \frac{\Log\left(\barx_{t}\exp(\bm{\epsilon}_{t}^{\wedge})\barx_{t}^{-1}\right)}{\bm{\epsilon}_{t}}\\
  & = \lim_{\bm{\epsilon}_{t}\rightarrow \bm{0}}
  \frac{\Log\left(\exp(\barx_{t}\bm{\epsilon}_{t}^{\wedge}\barx_{t}^{-1})\right)}{\bm{\epsilon}_{t}}\\
  & = \lim_{\bm{\epsilon}_{t}\rightarrow \bm{0}}
  \frac{\Log\left(\Exp(\textbf{Ad}_{\barx_{t}}\bm{\epsilon}_{t})\right)}{\bm{\epsilon}_{t}}\\
  & = \lim_{\bm{\epsilon}_{t}\rightarrow \bm{0}}
  \frac{\textbf{Ad}_{\barx_{t}}\bm{\epsilon}_{t}}{\bm{\epsilon}_{t}}\\
  & = \frac{\partial\textbf{Ad}_{\barx_{t}}\bm{\epsilon}_{t}}{\partial\bm{\epsilon}_{t}}\\
  & = \textbf{Ad}_{\barx_{t}}
\end{align*}

Therefore, the predicted state and error covariance are
\begin{align*}
  \checkx_{t} & = \barx_{t}=\barx_{t-1}\oplus \bm{u}_{t}\\
  \check{\bm{\Sigma}}_{t} & = \bm{F}_{t}\hat{\bm{\Sigma}}_{t-1}\bm{F}_{t}^{T}
  +\bm{W}_{t}\bm{Q}_{t}\bm{W}_{t}^{T}
\end{align*}

For update step:

Define the innovation $z_{t}$ such that
\begin{align*}
  \bm{z}_{t} & = \barx_{t}\cdot\left(\bm{y}_{t}-\bm{\bar{y}}_{t}\right)\\
  & = \barx_{t}\x_{t}^{-1}\cdot\bm{b}+\barx_{t}\cdot\bm{\delta}_{t}-\bm{b}
\end{align*}

\begin{align*}
  \bm{H}_{t} & =
  \left.\frac{\partial h}{\partial\x}\right|_{\x_{t}=\barx_{t}}\\
  & = \lim_{\x_{t} \rightarrow \barx_{t}}\frac{h(\x_{t})-h(\barx_{t})}
  {\x_{t} \ominus \barx_{t}}\\
  & = \lim_{\prescript{\mveps}{}{\bm{\xi}_{t}}\rightarrow \bm{0}}
  \frac{\Exp(-\prescript{\mveps}{}{\bm{\xi}_{t}})\cdot\bm{b}-\bm{b}}
  {\prescript{\x}{}{\bm{\xi}_{t}}}\\
  & = \lim_{\prescript{\mveps}{}{\bm{\xi}_{t}}\rightarrow \bm{0}}
  \frac{(\mathpzc{I}-\prescript{\mveps}{}{\bm{\xi}}_{t}^{\wedge})\cdot\bm{b}-\bm{b}}
  {\prescript{\mveps}{}{\bm{\xi}_{t}}}\\
  & = \lim_{\prescript{\mveps}{}{\bm{\xi}_{t}}\rightarrow\bm{0}}
  \frac{-\bm{\phi}^{\wedge}\bm{b}-\bm{\rho}}{\prescript{\x}{}{\bm{\xi}_{t}}}\\
  & = \lim_{\prescript{\mveps}{}{\bm{\xi}_{t}}\rightarrow\bm{0}}
  \frac{\bm{b}^{\wedge}\bm{\phi}-\bm{\rho}}{\prescript{\x}{}{\bm{\xi}_{t}}}\\
  & = \begin{bmatrix}
    \left.\frac{\partial(\bm{b}^{\wedge}\bm{\phi}-\bm{\rho})}
      {\partial\bm{\rho}}\right|_{\bm{\rho}=\bm{0}} &
    \left.\frac{\partial(\bm{b}^{\wedge}\bm{\phi}-\bm{\rho})}
      {\partial\bm{\phi}}\right|_{\bm{\phi}=\bm{0}}\end{bmatrix}\\
  & = \begin{bmatrix}
    -\mathpzc{I} & [\bm{b}]_{\times}
  \end{bmatrix}
\end{align*}
and $\bm{V}_{t}=\bar{\bm{R}}_{t}^{T}$.

Furthermore, we have $\Exp(\prescript{\mveps}{}{\bm{\xi}_{t}})=
\x_{t}\barx_{t}^{-1}$, which implies that
$\x_{t}=\Exp(\prescript{\mveps}{}{\bm{\xi}_{t}})\barx_{t}$. This suggest the update 
of $\x_{t}$ is
\[
  \hatx_{t}=\Exp(\bm{K}_{t}\bm{z}_{t})\checkx_{t},
\]
where $\bm{K}_{t}=\check{\bm{\Sigma}}_{t}\bm{H}_{t}\bm{S}_{t}^{-1}$ and
$\bm{S}_{t}=\bm{H}_{t}\check{\bm{\Sigma}}_{t}\bm{H}_{t}^{T}
+\bm{V}_{t}\bm{R}_{t}\bm{V}_{t}^{T}$.

Thus, the updated state and error covariance are given by:
\begin{align*}
  \hatx_{t} & = \Exp(\bm{K}_{t}\bm{z}_{t})\checkx_{t}\\
  \hat{\bm{\Sigma}}_{t} & =
  (\mathpzc{I}-\bm{K}_{t}\bm{H}_{t})\check{\bm{\Sigma}}_{t}
\end{align*}
\subsection{Example: $SE_{2}(3)$}
\subsection{Example: $SE_{k}(3)$}

\section{Practical Implications of the ESKF on Matrix Lie Groups}
\subsection{Localization for Autonomous Driving: $SE(3)$ or $SE_{2}(3)$?}
\subsection{Drone Navigation?}

\section{Conclusion}
\section{Appendix}
\addcontentsline{toc}{subsection}{[put $SE(3)$, $SE_{2}(3)$, and $SE_{k}(3)$
property here]}
\addcontentsline{toc}{subsection}{[put Jacobians derivation here]}

\bibliographystyle{apalike}
\bibliography{eskf_lie}

\end{document}
